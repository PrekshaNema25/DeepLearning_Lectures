%Artifical Neuron
\begin{tikzpicture}
	\node[name=s,shape=circle split,draw=gray!40,line width=0mm,minimum size=2cm] {};

	\begin{scope}[on background layer]
		\fill[black!50] (s.base) ([xshift=-0mm]s.east) arc (0:180:1cm-0mm)--cycle;
		\fill[white!50] (s.base) ([xshift=0mm]s.west) arc (180:360:1cm-0mm)--cycle;
	\end{scope}

	\onslide<2->{\node (input0) at (-1.5, -2.0) {$x_1$};}
	\onslide<3->{\node (input1) at (-0.75, -2.0) {$x_2$};}
	\onslide<4->{\node (input2) at (0, -2.0) {$..$};}
	\onslide<4->{\node (input3) at (0.75, -2.0) {$..$};}
	\onslide<5->{\node (input4) at (2.25, -2.0) {$x_n \onslide<6->{\in \{0,1\}}$};}

	\onslide<9->{\node (output) at (0, 2.0) {$y \in \{0,1\}$};}

	\onslide<7->{\node at (0,-0.5) {$g$};}
	\onslide<8->{\node at (0,0.5) {$f$};}

	\draw [->] (input0) -- (s.240);
	\draw [->] (input1) -- (s.255);
	\draw [->] (input2) -- (s.270);
	\draw [->] (input3) -- (s.285);
	\draw [->] (input4) -- (s.300);
	\draw [->] (s.90) -- (output);

\end{tikzpicture}