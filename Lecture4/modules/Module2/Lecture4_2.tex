\savestack{\nn}{% \hspace{-0.1in}
\tikzstyle{input_neuron}=[circle,draw=red!50,fill=red!10,thick,minimum size=8mm]
\tikzstyle{hidden_neuron}=[circle,draw=blue!50,fill=cyan!10,thick,minimum size=8mm]
\tikzstyle{output_neuron}=[circle,draw=green!50,fill=green!10,thick,minimum size=8mm]
\tikzstyle{bias_neuron}=[circle,draw=red!50,fill=red!10,thick,minimum size=4mm]
\tikzstyle{bias_hidden_neuron}=[circle,draw=blue!50,fill=cyan!10,thick,minimum size=4mm]
\tikzstyle{input}=[circle,draw=black!50,fill=black!20,thick,minimum size=8mm]
\begin{tikzpicture}
	\node [input_neuron] (neuron01) at (0,0) {$x_1$};
	\node [input_neuron] (neuron02) at (2,0){$x_2$};
	\node [input_neuron] (neuron03) at (4,0) {$x_n$};

	\node [bias_neuron] (neuron04) at (5.2,0.4) {};

	\node [hidden_neuron] (neuron11) at (0,2)  {};
	\node [hidden_neuron] (neuron12) at (2,2)  {};
	\node [hidden_neuron] (neuron13) at (4,2)  {};

	\node [bias_hidden_neuron] (neuron14) at (5.2,2.4) {};

	\begin{scope}
		\path[clip] (0,2) circle (4mm);
		\path[fill=blue!50] (-0.4,2) rectangle (0.4,2.4);
	\end{scope}

	\begin{scope}
		\path[clip] (2,2) circle (4mm);
		\path[fill=blue!50] (1.6,2) rectangle (2.4,2.4);
	\end{scope}

	\begin{scope}
		\path[clip] (4,2) circle (4mm);
		\path[fill=blue!50] (3.6,2) rectangle (4.4,2.4);
	\end{scope}

	\node [hidden_neuron] (neuron21) at (0,4)  {};
	\node [hidden_neuron] (neuron22) at (2,4)  {};
	\node [hidden_neuron] (neuron23) at (4,4)  {};

	\node [bias_hidden_neuron] (neuron24) at (5.2,4.4) {};

	\begin{scope}
		\path[clip] (0,4) circle (4mm);
		\path[fill=blue!50] (-0.4,4) rectangle (0.4,4.4);
	\end{scope}

	\begin{scope}
		\path[clip] (2,4) circle (4mm);
		\path[fill=blue!50] (1.6,4) rectangle (2.4,4.4);
	\end{scope}
	\begin{scope}
		\path[clip] (4,4) circle (4mm);
		\path[fill=blue!50] (3.6,4) rectangle (4.4,4.4);
	\end{scope}

	\node [output_neuron] (neuron31) at (1,6)  {};
	\node [output_neuron] (neuron32) at (3,6)  {};

	\begin{scope}
		\path[clip] (1,6) circle (4mm);
		\path[fill=green!50] (0.6,6) rectangle (1.4,6.4);
	\end{scope}

	\begin{scope}
		\path[clip] (3,6) circle (4mm);
		\path[fill=green!50] (2.6,6) rectangle (3.4,6.4);
	\end{scope}

	\draw[white,->] (neuron01) -- (neuron11) node[black,pos=.5,right]  {$W_{1}$} node[black,pos=0.8,left] {$a_{1}$};

	\draw[white,->] (neuron11) -- (neuron21) node[black,pos=.5,right] {$W_{2}$} node[black,pos=0.8,left] {$a_{2}$} node[black,pos=.2,left] {$h_{1}$};
	\draw[white,->] (neuron21) -- (neuron31) node[black,pos=.5,right] {$W_{3}$} node[black,pos=0.8,left] {$a_{3}$} node[black,pos=.2,left] {$h_{2}$};
	\draw[white,->] (neuron04) -- (neuron13) node[black,pos=0,right,above] {$b_1$};
	\draw[white,->] (neuron14) -- (neuron23) node[black,pos=0,right,above] {$b_2$};
	\draw[white,->] (neuron24) -- (neuron32) node[black,pos=0,right,above] {$b_3$};

	\draw[white,->] (neuron31) -- (1,6.5) node[black,pos=1,above] {$h_L = \hat{y} = f(x)$};

	\draw[black!20,line width=2pt,loosely dotted] (neuron01) -- (neuron02);
	\draw[black!20,line width=2pt,loosely dotted] (neuron02) -- (neuron03);
	\draw[black!20,line width=2pt,loosely dotted] (neuron11) -- (neuron12);
	\draw[black!20,line width=2pt,loosely dotted] (neuron12) -- (neuron13);
	\draw[black!20,line width=2pt,loosely dotted] (neuron21) -- (neuron22);
	\draw[black!20,line width=2pt,loosely dotted] (neuron22) -- (neuron23);
	\draw[black!20,line width=2pt,loosely dotted] (neuron31) -- (neuron32);

	\foreach \from in {neuron01,neuron02,neuron03,neuron04}
	\foreach \to in {neuron11,neuron12,neuron13}
	\draw [black!50,line width=1.5pt,->] (\from) -- (\to);

	\foreach \from in {neuron11,neuron12,neuron13,neuron14}
	\foreach \to in {neuron21,neuron22,neuron23}
	\draw [black!50,line width=1.5pt,->] (\from) -- (\to);

	\foreach \from in {neuron21,neuron22,neuron23,neuron24}
	\foreach \to in {neuron31,neuron32}
	\draw [black!50,line width=1.5pt,->] (\from) -- (\to);
\end{tikzpicture}}

\begin{frame}
  \myheading{Module 4.2: Learning Parameters of Feedforward Neural Networks (Intuition)}
\end{frame}

%Slide 08
\begin{frame}
  \begin{block}{The story so far...}
    \begin{itemize}
      \justifying
      \item We have introduced feedforward neural networks
      \item We are now interested in finding an algorithm for learning the parameters of this model
    \end{itemize}
  \end{block}
\end{frame}

%Slide 09
\begin{frame}
  \begin{columns}
    \column{0.4\textwidth}
    \begin{overlayarea}{\textwidth}{\textheight}
      \makebox[\textwidth][c]{\usebox{\nncontent}}
    \end{overlayarea}

    \column{0.6\textwidth}
    \begin{overlayarea}{\textwidth}{\textheight}
      \begin{itemize}
        \justifying
        \item Recall our gradient descent algorithm
        \item \visible<3-> {We can write it more concisely as}
            \visible<2-> {
              \begin{algorithm}[H]
                \SetAlgoLined
                $t \leftarrow 0$\;
                $max\_iterations\leftarrow 1000$\;
                \only<2-3>{$Initialize \quad w_0, b_0$\;}
                \only<4-7>{$Initialize \quad \theta_0 = [w_0, b_0]$\;}
                \only<8>{$Initialize \quad \alert<8>{\theta_0 = [W_1^{0}, ..., W_{L}^{0}, b_1^{0}, ..., b_{L}^{0}]}$\;}
                \While{$t\texttt{++} < max\_iterations$}{
                  \only<2-3>{$w_{t+1} \leftarrow w_{t} - \eta \nabla w_{t}$\;}
                  \only<2-3>{$b_{t+1} \leftarrow b_{t} - \eta \nabla b_{t}$\;}
                  \only<4->{$\theta_{t+1} \leftarrow \theta_{t} - \eta \nabla \theta_{t}$\;}
                }
                \caption{gradient\_descent()}
              \end{algorithm}
            }
        \item<5-> where $\nabla \theta_{t} = \big[\frac{\partial \mathscr{L}(\theta)}{\partial w_t}, \frac{\partial \mathscr{L}(\theta)}{\partial b_t}\big]^{T}$
        \item<6-> Now, in this feedforward neural network, instead of $\theta = [w, b]$ we have $\theta = {W_{1}, W_2, .., W_{L}, b_1, b_2, .., b_{L}}$
        \item<7-> We can still use the same algorithm for learning the parameters of our model

      \end{itemize}
    \end{overlayarea}

  \end{columns}
\end{frame}

%Slide 10
\begin{frame}
  \begin{overlayarea}{\textwidth}{\textheight}
    \begin{itemize}
      \justifying
      \item<1-> Except that now our $\nabla \theta$ looks much more nasty \hspace{5cm}
          % \vspace{0.2cm}
  
          \visible<2->{
             {
            \begin{align*}
            \hspace{-1cm}
            \begingroup
            \setlength\arraycolsep{2pt}
            \begin{bmatrix}
                  \visible<2->{\frac{\partial \mathscr{L}(\theta)}{\partial W_{111}}} 
                    & \visible<3->{\dots}  
                    & \visible<4->{\frac{\partial \mathscr{L}(\theta)}{\partial W_{11n}}}
                    & \visible<6->{\frac{\partial \mathscr{L}(\theta)}{\partial W_{211}} }
                    & \visible<6->{\dots}
                    & \visible<6->{\frac{\partial \mathscr{L}(\theta)}{\partial W_{21n}} }
                    & \visible<7->{\dots} 
                    & \visible<8->{\frac{\partial \mathscr{L}(\theta)}{\partial W_{L,11}} }
                    & \visible<8->{\dots} 
                    & \visible<8->{\frac{\partial \mathscr{L}(\theta)}{\partial W_{L,1k}} }
                    & \visible<8->{\frac{\partial \mathscr{L}(\theta)}{\partial W_{L,1k}} }
                    & \visible<9->{\frac{\partial \mathscr{L}(\theta)}{\partial b_{11}} }
                    & \visible<9->{\dots} 
                    & \visible<9->{\frac{\partial \mathscr{L}(\theta)}{\partial b_{L1}} }
                    \\
                   & & & & & & & & & & & & & 
                    \\
                  \visible<5->{ \frac{\partial \mathscr{L}(\theta)}{\partial W_{121}} }
                    & \visible<5->{ \dots }
                    & \visible<5->{ \frac{\partial \mathscr{L}(\theta)}{\partial W_{12n}} }
                    & \visible<6->{\frac{\partial \mathscr{L}(\theta)}{\partial W_{221}}}
                    & \visible<6->{ \dots}
                    & \visible<6->{\frac{\partial \mathscr{L}(\theta)}{\partial W_{22n}}}
                    & \visible<7->{\dots} 
                    & \visible<8->{\frac{\partial \mathscr{L}(\theta)}{\partial W_{L,21}}}
                    & \visible<8->{\dots}
                    & \visible<8->{\frac{\partial \mathscr{L}(\theta)}{\partial W_{L,2k}}}
                    & \visible<8->{\frac{\partial \mathscr{L}(\theta)}{\partial W_{L,2k}}}
                    & \visible<9->{\frac{\partial \mathscr{L}(\theta)}{\partial b_{12}} }
                    & \visible<9->{\dots}
                    & \visible<9->{\frac{\partial \mathscr{L}(\theta)}{\partial b_{L2}} }
                    \\
                  \visible<5->{\vdots} 
                    & \visible<5->{\vdots}  
                    & \visible<5->{\vdots}  
                    & \visible<6->{\vdots} 
                    & \visible<6->{\vdots} 
                    & \visible<6->{\vdots} 
                    & \visible<7->{\vdots} 
                    & \visible<8->{\vdots} 
                    & \visible<8->{\vdots} 
                    & \visible<8->{\vdots} 
                    & \visible<8->{\vdots} 
                    & \visible<9->{\vdots} 
                    & \visible<9->{\vdots} 
                    & \visible<9->{\vdots}
                    \\
                  \visible<5->{\frac{\partial \mathscr{L}(\theta)}{\partial W_{1n1}} }  
                    & \visible<5->{\dots}   
                    & \visible<5->{\frac{\partial \mathscr{L}(\theta)}{\partial W_{1nn}}}   
                    & \visible<6->{\frac{\partial \mathscr{L}(\theta)}{\partial W_{2n1}}}  
                    & \visible<6->{\dots}  
                    & \visible<6->{\frac{\partial \mathscr{L}(\theta)}{\partial W_{2nn}}}  
                    & \visible<7->{\dots}  
                    & \visible<8->{\frac{\partial \mathscr{L}(\theta)}{\partial W_{L,n1}}}  
                    & \visible<8->{\dots } 
                    & \visible<8->{\frac{\partial \mathscr{L}(\theta)}{\partial W_{L,nk}} } 
                    & \visible<8->{\frac{\partial \mathscr{L}(\theta)}{\partial W_{L,nk}} } 
                    & \visible<9->{\frac{\partial \mathscr{L}(\theta)}{\partial b_{1n}} } 
                    & \visible<9->{\dots} 
                    & \visible<9->{\frac{\partial \mathscr{L}(\theta)}{\partial b_{Ln}} } 
                    %\\
                \end{bmatrix}
                \endgroup
                \end{align*}
              }
          }
          \vspace{0.2in}
      % \item<9-> ... and similar entries for partial derivatives w.r.t. the elements of $b_1, b_2, ..., b_{L}$

      \item<10-> $\nabla \theta$ is thus composed of \\
          $\nabla W_1, \nabla W_2, ... \nabla W_{L-1}\in \mathbb{R}^{n \times n}, \nabla W_{L} \in \mathbb{R}^{n \times k},$ \\
          $\nabla b_1, \nabla b_2, ..., \nabla b_{L-1} \in \mathbb{R}^n $ and $\nabla b_{L} \in \mathbb{R}^k$
    \end{itemize}
  \end{overlayarea}
\end{frame}

%Slide 11
\begin{frame}
  \begin{block}{We need to answer two questions}
    \begin{itemize}
      \justifying
      \item<2-> How to choose the loss function $\mathscr{L}(\theta)$?
      \item<3-> How to compute $\nabla \theta$ which is composed of $\nabla W_1, \nabla W_2, ..., \nabla W_{L-1} \in \mathbb{R}^{n \times n}, \nabla W_L \in \mathbb{R}^{n \times k},$ \\ $\nabla b_1, \nabla b_2, ..., \nabla b_{L-1} \in \mathbb{R}^n $ and $\nabla b_L \in \mathbb{R}^k$ ?
    \end{itemize}
  \end{block}
\end{frame}

