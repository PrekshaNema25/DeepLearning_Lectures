\begin{frame}
	\myheading{Module 10.10: Relation between SVD \& word2Vec}
\end{frame}

\begin{frame}
	\begin{block}{The story ahead ...}
		\onslide<1->{
			\begin{itemize}\justifying
				\item Continuous bag of words model
				\item Skip gram model with negative sampling (the famous word2vec)
				\item GloVe word embeddings
				\item Evaluating word embeddings
				\item \textcolor<1->{red}{Good old SVD does just fine!!}
			\end{itemize}}
	\end{block}
\end{frame}


\begin{frame}
	\begin{columns}
		\column{0.5\textwidth}
		\begin{overlayarea}{\textwidth}{\textheight}
			\begin{tikzpicture}

\node (R)[text width=\textwidth] at (0,3) {\footnotesize{\begin{table}
\begin{tabular}{|c|c|c|c|c|c|c|}
\hline
0 & 0 & 1 & ... & 0 & 0 & 0\\
\hline
\end{tabular}
\end{table}}
};
\node (H) [text width=\textwidth] at ($ (R) + (0,2) $) {\begin{table}
\begin{tabular}{l!{\color{white}\vrule}l!{\color{white}\vrule}l!{\color{white}\vrule}l!{\color{white}\vrule}l!{\color{white}\vrule}l!{\color{white}\vrule}l!{\color{white}\vrule}l!{\color{white}\vrule}l!{\color{white}\vrule}l}
\hline
\rowcolor{blue!50}
 .&.&.&.&.&.&.&.&.&.\\
\hline
\end{tabular}
\end{table}
};

\draw [fill=red!50] ($(H) + (-3.75,1.5)$) rectangle ($ (H) + (-2.25, 1.8)$);
\draw [fill=red!50] ($(H) + (-1.75,1.5)$) rectangle ($ (H) + (-0.25, 1.8)$);
\draw [fill=red!50] ($(H) + (0.25,1.5)$) rectangle ($ (H) + (1.75, 1.8)$);
\draw [fill=red!50] ($(H) + (2.25,1.5)$) rectangle ($ (H) + (3.75, 1.8)$);
\node [color=red!50] at ($ (H) + (-3,2)$) {\textit{he}};
\node[color=red!50] at ($ (H) + (-1,2)$) {\textit{sat}};
\node[color=red!50] at ($(H) + (1,2)$) {\textit{a}};
\node[color=red!50] at ($(H) + (3,2)$) {\textit{chair}};
\if 0
\node(V) [text width=\textwidth] at ($ (H) + (0,1.5) $) {\large{\begin{table}
\begin{tabular}{l!{\color{white}\vrule}l!{\color{white}\vrule}l!{\color{white}\vrule}l!{\color{white}\vrule}l!{\color{white}\vrule}l!{\color{white}\vrule}l!{\color{white}\vrule}l!{\color{white}\vrule}l!{\color{white}\vrule}l!{\color{white}\vrule}l!{\color{white}\vrule}l}
\hline
\rowcolor{red!50}
 .&.&.&.&.&.&.&.&.&.&.&.\\
\hline
\end{tabular}
\end{table}}
};
 
\node[text width=0.15\textwidth, rotate=90] at ($(V) + (-3,0.9)$ ) {\tiny{$P(he|sat)$}};
\node[text width=0.15\textwidth, rotate=90] at ($(V) + (-2.5,0.9)$ ) {\tiny{$P(chair|sat)$}};
\node[text width=0.15\textwidth, rotate=90] at ($(V) + (-2.0,0.9)$ ) {\tiny{$P(man|sat)$}};
\node[text width=0.15\textwidth, rotate=90] at ($(V) + (1.4,0.9)$ ) {\tiny\textcolor{red}{{$P(on|sat)$}}};
\fi

%\draw[line width=1][->] (0,1.6) -- (0,2.75);
\node[text width=0.3\textwidth, ] at ($(H) + (3.8,0)$ ) {\tiny{$\mathbf{h} \in \mathbb{R}^{|k|}$}};
\node[text width=0.3\textwidth, ] at ($(H) + (2.5,0.7)$ ) {\tiny{$W_{context} \in \mathbb{R}^{k\times |V|}$}};
\node[text width=0.3\textwidth, ] at ($(0,2.7) + (3.3,0.4)$ ) {\tiny{$\mathbf{x} \in \mathbb{R}^{|V|}$}};
\node[text width = 0.3\textwidth] at (0.7, 4){\tiny{$W_{word} \in \mathbb{R}^{k\times |V|}$}};
\draw[line width=0.5] (-2.15,3.15) -- (-2.7,4.7);
\draw[line width=0.5] (2.15,3.15) -- (2.7,4.7);

\draw[line width=0.5] [->](0,5.20) -- ($(H) + (-3,1.5)$);
\draw[line width=0.5] [->] (0,5.20) -- ($(H) + (3,1.5)$);
\draw[line width=0.5] [->] (0,5.20) -- ($(H) + (-1,1.5)$);
\draw[line width=0.5] [->] (0,5.20) -- ($(H) + (1,1.5)$);
\end{tikzpicture}

		\end{overlayarea}
		\column{0.5\textwidth}
		\begin{overlayarea}{\textwidth}{\textheight}
			\footnotesize{\begin{itemize} \justifying
					\onslide<1->{\item Recall that SVD does a matrix factorization of the co-occurrence matrix}
					      \onslide<2->{\item Levy et.al [2015] show that word2vec also implicitly does a matrix factorization}
					      \onslide<3->{\item What does this mean ?}
					      \onslide<4->{\item Recall that word2vec gives us $W_{context}$  $\&$  $W_{word}$ }.
					      \onslide<5->{\item Turns out that we can also show that
					      \vspace{-0.1in}
					      \begin{align*}
						      M = W_{context}*W_{word}
					      \end{align*}
					      where
					      \begin{align*}
						      M_{ij} & = PMI(w_i,c_i)-log(k)               \\
						      k      & = \text{number of negative samples}
					      \end{align*}
					      }
					      \vspace{-0.1in}
					      \onslide<6->{\item So essentially, word2vec factorizes a matrix M which is related to the PMI based co-occurrence matrix (very similar to what SVD does)}

				\end{itemize}}
		\end{overlayarea}
	\end{columns}
\end{frame}
