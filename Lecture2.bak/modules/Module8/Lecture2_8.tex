\begin{frame}
	\myheading{Module 2.8: Representation Power of a Network of Perceptrons}
\end{frame}

\begin{frame}
	\begin{itemize}\justifying
		\item<1-> We will now see how to implement \textbf{any} boolean function using a network of perceptrons ...
	\end{itemize}
\end{frame}


\begin{frame}
	\begin{columns}
		\column{0.45\textwidth}
		\begin{overlayarea}{\textwidth}{\textheight}
			\vspace{1cm}
			\begin{tikzpicture}[scale=0.9,every node/.style={transform shape}]
	\onslide<10->{\node [output_neuron] (out0) at (-3,12)  {} ;}

	\onslide<2->{\node (input0) at (-4,8)  {$x_1$};}
	\onslide<2->{\node (input1) at (-2,8) {$x_2$};}

	\onslide<2->{\node [hidden_neuron] (hi0) at (-6,10)  {  } ;}
	\onslide<2->{\node [hidden_neuron] (hi1) at (-4,10)  {  } ;}
	\onslide<2->{\node [hidden_neuron] (hi2) at (-2,10)  {  } ;}
	\onslide<2->{\node [hidden_neuron] (hi3) at (0,10)  {  } ;}



	\onslide<9->{\node (bias) at (-6,9)  {$bias=$-2} ;}

	\onslide<13->{\node (output0) at (-3,13)  {$y$} ;}

	\onslide<12->{\node (w1) at (-5,11)  {$w_1$} ;}
	\onslide<12->{\node (w2) at (-3.8,11)  {$w_2$} ;}
	\onslide<12->{\node (w3) at (-2.8,11)  {$w_3$} ;}
	\onslide<12->{\node (w4) at (-1.85,11)  {$w_4$} ;}

	\onslide<5->{\draw [->, thick, red] (input0) -- (hi0);}
	\onslide<5->{\draw [->, thick, red] (input1) -- (hi0);}

	\onslide<6->{\draw [->, thick, red](input0) -- (hi1);}
	\onslide<6->{\draw [->, thick, blue](input1) -- (hi1);}

	\onslide<7->{\draw [->, thick, blue] (input0) -- (hi2);}
	\onslide<7->{\draw [->, thick, red] (input1) -- (hi2);}

	\onslide<8->{\draw [->, thick, blue](input0) -- (hi3);}
	\onslide<8->{\draw [->, thick, blue](input1) -- (hi3);}

	\onslide<11->{\draw [->] (hi0) -- (out0);}
	\onslide<11->{\draw [->] (hi1) -- (out0);}
	\onslide<11->{\draw [->] (hi2) -- (out0);}
	\onslide<11->{\draw [->] (hi3) -- (out0);}

	\onslide<13->{\draw [->] (out0) -- (output0);}

	%\node at (-6,1) {red edge indicates $w$ = -1};
	%\node at (-6,0) {green edge indicates $w$ = +1};

\end{tikzpicture}
			\onslide<4->{red edge indicates $w$ = -1 \\}
			\onslide<4->{blue edge indicates $w$ = +1}
		\end{overlayarea}
		\column{0.55\textwidth}
		\begin{overlayarea}{\textwidth}{\textheight}
			\only<1-13>{
				\begin{itemize}\justifying
					\item For this discussion, we will assume True = +1 and False = -1
					\item<2-> We consider 2 inputs and 4 perceptrons
					\item<3-> Each input is connected to all the 4 perceptrons with specific weights
					\item<9-> The bias ($w_0$) of each perceptron is -2 (i.e., each perceptron will fire only if the weighted sum of its input is $\geq$ 2)
					\item<10-> Each of these perceptrons is connected to an output perceptron by weights (which need to be learned)
					\item<13> The output of this perceptron ($y$) is the output of this network
				\end{itemize}
			}
		\end{overlayarea}
	\end{columns}
\end{frame}


\begin{frame}
	\begin{columns}
		\column{0.45\textwidth}
		\begin{overlayarea}{\textwidth}{\textheight}
			\vspace{1cm}
			\begin{tikzpicture}
	\node (input0) at (8,-0.1)  {$x_{1}$};
	\node (input1) at (9,-0.1)  {$x_{2}$};
	\node (input2) at (10,-0.1)  {$..$};
	\node (input3) at (11,-0.1)  {$..$};
	\node (input4) at (12,-0.1)  {$x_{n}$};
	\node (input5) at (7,-0.1)  {$x_{0}=1$};

	\node [hidden_neuron] (neuron1) at (10,2)  {};


	\node (output0)  at (10,3.5) {$y$};

	\draw [->] (input0) -- (neuron1);
	\draw [->] (input1) -- (neuron1);
	\draw [->] (input2) -- (neuron1);
	\draw [->] (input3) -- (neuron1);
	\draw [->] (input4) -- (neuron1);
	\draw [->] (input5) -- (neuron1);

	\draw [->] (neuron1) -- (output0);

	\node (formula)[scale=.8] at (8.4,0.6) {$w_{1}$};
	\node (formula)[scale=.8] at (9.1,0.6) {$w_{2}$};
	\node (formula)[scale=.8] at (9.8,0.6) {$..$};
	\node (formula)[scale=.8] at (10.4,0.6) {$..$};
	\node (formula)[scale=.8] at (11.1,0.6) {$w_{n}$};
	\node (formula)[scale=.8] at (7.2,0.6) {$w_{0} = -\theta$};

\end{tikzpicture}

			red edge indicates $w$ = -1 \\
			blue edge indicates $w$ = +1
		\end{overlayarea}
		\column{0.55\textwidth}
		\begin{overlayarea}{\textwidth}{\textheight}
			\textbf{Terminology:}
			\begin{itemize}\justifying
				\item This network contains 3 layers
				\item<2-> The layer containing the inputs ($x_1, x_2$) is called the \textbf{input layer}
				\item<3-> The middle layer containing the 4 perceptrons is called the \textbf{hidden layer}
				\item<4-> The final layer containing one output neuron is called the \textbf{output layer}
				\item<5-> The outputs of the 4 perceptrons in the hidden layer are denoted by $h_1, h_2, h_3, h_4$
				\item<6-> The red and blue edges are called layer 1 weights
				\item<7-> $w_1, w_2, w_3, w_4$ are called layer 2 weights
			\end{itemize}
		\end{overlayarea}
	\end{columns}
\end{frame}


\begin{frame}
	\begin{columns}
		\column{0.45\textwidth}
		\begin{overlayarea}{\textwidth}{\textheight}
			\vspace{1cm}
			\begin{tikzpicture}[scale=0.9,every node/.style={transform shape}]
	\node [output_neuron] (out0) at (-3,12)  {} ;

	\node (input0) at (-4,8)  {$x_1$};
	\node (input1) at (-2,8) {$x_2$};

	\node [hidden_neuron] (hi0) at (-6,10)  {  } ;
	\node [hidden_neuron] (hi1) at (-4,10)  {  } ;
	\node [hidden_neuron] (hi2) at (-2,10)  {  } ;
	\node [hidden_neuron] (hi3) at (0,10)  {  } ;

	\node (h1_label) at (-6,10.75)  { $h_1$ } ;
	\node (h2_label) at (-4,10.75)  { $h_2$ } ;
	\node (h3_label) at (-2,10.75)  { $h_3$ } ;
	\node (h4_label) at (-0,10.75)  { $h_4$ } ;

	\onslide<6->{\node [hidden_neuron] (hi0) at (-6,10)  { -1,-1 } ;}
	\onslide<7->{\node [hidden_neuron] (hi1) at (-4,10)  { -1,1 } ;}
	\onslide<8->{\node [hidden_neuron] (hi2) at (-2,10)  { 1,-1 } ;}
	\onslide<9->{\node [hidden_neuron] (hi3) at (0,10)  { 1,1 } ;}

	\node (bias) at (-6,9)  {$bias=$-2} ;

	\node (output0) at (-3,13)  {$y$} ;

	\node (w1) at (-5,11)  {$w_1$} ;
	\node (w2) at (-3.8,11)  {$w_2$} ;
	\node (w3) at (-2.8,11)  {$w_3$} ;
	\node (w4) at (-1.85,11)  {$w_4$} ;

	\draw [->, thick, red] (input0) -- (hi0);
	\draw [->, thick, red] (input1) -- (hi0);

	\draw [->, thick, red](input0) -- (hi1);
	\draw [->, thick, blue](input1) -- (hi1);

	\draw [->, thick, blue] (input0) -- (hi2);
	\draw [->, thick, red] (input1) -- (hi2);

	\draw [->, thick, blue](input0) -- (hi3);
	\draw [->, thick, blue](input1) -- (hi3);

	\draw [->] (hi0) -- (out0);
	\draw [->] (hi1) -- (out0);
	\draw [->] (hi2) -- (out0);
	\draw [->] (hi3) -- (out0);

	\draw [->] (out0) -- (output0);

\end{tikzpicture}
			red edge indicates $w$ = -1 \\
			blue edge indicates $w$ = +1
		\end{overlayarea}
		\column{0.55\textwidth}
		\begin{overlayarea}{\textwidth}{\textheight}
			\only<1->{
				\begin{itemize}\justifying
					\item We claim that this network can be used to implement \textbf{any} boolean function (linearly separable or not) !
					\item<2-> In other words, we can find $w_1, w_2, w_3, w_4$ such that the truth table of any boolean function can be represented by this network
					\item<3-> Astonishing claim! \onslide<4->{Well, not really, if you understand what is going on}
					\item<5-> Each perceptron in the middle layer fires only for a specific input (and no two perceptrons fire for the same input)
					      \only<6>{\item the first perceptron fires for \{-1,-1\} }
					      \only<7>{\item the second perceptron fires for \{-1,1\} }
					      \only<8>{\item the third perceptron fires for \{1,-1\} }
					      \only<9>{\item the fourth perceptron fires for \{1,1\} }
					\item<10-> Let us see why this network works by taking an example of the XOR function
				\end{itemize}
			}
		\end{overlayarea}
	\end{columns}
\end{frame}


\begin{frame}
	\begin{columns}
		\column{0.45\textwidth}
		\begin{overlayarea}{\textwidth}{\textheight}
			\vspace{1cm}
			\begin{tikzpicture}[scale=0.9,every node/.style={transform shape}]
	\node [output_neuron] (out0) at (-3,12)  {} ;

	\node (input0) at (-4,8)  {$x_1$};
	\node (input1) at (-2,8) {$x_2$};

	\node [hidden_neuron] (hi0) at (-6,10)  {  } ;
	\node [hidden_neuron] (hi1) at (-4,10)  {  } ;
	\node [hidden_neuron] (hi2) at (-2,10)  {  } ;
	\node [hidden_neuron] (hi3) at (0,10)  {  } ;

	\node (h1_label) at (-6,10.75)  { $h_1$ } ;
	\node (h2_label) at (-4,10.75)  { $h_2$ } ;
	\node (h3_label) at (-2,10.75)  { $h_3$ } ;
	\node (h4_label) at (-0,10.75)  { $h_4$ } ;

	\node [hidden_neuron] (hi0) at (-6,10)  { -1,-1 } ;
	\node [hidden_neuron] (hi1) at (-4,10)  { -1,1 } ;
	\node [hidden_neuron] (hi2) at (-2,10)  { 1,-1 } ;
	\node [hidden_neuron] (hi3) at (0,10)  { 1,1 } ;


	\node (bias) at (-6,9)  {$bias=$-2} ;

	\node (output0) at (-3,13)  {$y$} ;

	\node (w1) at (-5,11)  {$w_1$} ;
	\node (w2) at (-3.8,11)  {$w_2$} ;
	\node (w3) at (-2.8,11)  {$w_3$} ;
	\node (w4) at (-1.85,11)  {$w_4$} ;

	\draw [->, thick, red] (input0) -- (hi0);
	\draw [->, thick, red] (input1) -- (hi0);

	\draw [->, thick, red](input0) -- (hi1);
	\draw [->, thick, blue](input1) -- (hi1);

	\draw [->, thick, blue] (input0) -- (hi2);
	\draw [->, thick, red] (input1) -- (hi2);

	\draw [->, thick, blue](input0) -- (hi3);
	\draw [->, thick, blue](input1) -- (hi3);

	\draw [->] (hi0) -- (out0);
	\draw [->] (hi1) -- (out0);
	\draw [->] (hi2) -- (out0);
	\draw [->] (hi3) -- (out0);

	\draw [->] (out0) -- (output0);
\end{tikzpicture}

			red edge indicates $w$ = -1 \\
			blue edge indicates $w$ = +1
		\end{overlayarea}
		\column{0.55\textwidth}
		\begin{overlayarea}{\textwidth}{\textheight}
			\begin{itemize}\justifying
				\item Let $w_0$ be the bias output of the neuron (\textit{i.e.}, it will fire if $\sum_{i=1}^{4}w_ih_i \geq w_0$)
			\end{itemize}
			\onslide<2->{
				\small{
					\begin{table}
						\begin{tabular}{cccccccc}
							\hline
							\onslide<2->{$x_1$} & \onslide<2->{$x_2$} & \onslide<2->{$XOR$} & \onslide<2->{$h_1$} & \onslide<2->{$h_2$} & \onslide<2->{$h_3$} & \onslide<2->{$h_4$} & \onslide<2->{$\sum_{i=1}^{4}w_ih_i$} \\
							\hline
							\onslide<2->{0}     & \onslide<2->{0}     & \onslide<2->{0}     & \onslide<2->{1}     & \onslide<2->{0}     & \onslide<2->{0}     & \onslide<2->{0}     & \onslide<2->{$w_1$}                  \\
							\onslide<3->{0}     & \onslide<3->{1}     & \onslide<3->{1}     & \onslide<3->{0}     & \onslide<3->{1}     & \onslide<3->{0}     & \onslide<3->{0}     & \onslide<3->{$w_2$}                  \\
							\onslide<4->{1}     & \onslide<4->{0}     & \onslide<4->{1}     & \onslide<4->{0}     & \onslide<4->{0}     & \onslide<4->{1}     & \onslide<4->{0}     & \onslide<4->{$w_3$}                  \\
							\onslide<5->{1}     & \onslide<5->{1}     & \onslide<5->{0}     & \onslide<5->{0}     & \onslide<5->{0}     & \onslide<5->{0}     & \onslide<5->{1}     & \onslide<5->{$w_4$}                  \\
							\hline
						\end{tabular}
					\end{table}
				}
			}
			\vspace{-0.2in}
			\textnormal{
				\begin{itemize}\justifying
					\item<6-> This results in the following four conditions to implement XOR: $w_1 < w_0, w_2 \geq w_0, w_3 \geq w_0, w_4 < w_0$
					\item<7-> Unlike before, there are no contradictions now and the system of inequalities can be satisfied
					\item<8-> Essentially each $w_i$ is now responsible for one of the 4 possible inputs and can be adjusted to get the desired output for that input
				\end{itemize}
			}
		\end{overlayarea}
	\end{columns}
\end{frame}

\begin{frame}
	\begin{columns}
		\column{0.45\textwidth}
		\begin{overlayarea}{\textwidth}{\textheight}
			\vspace{1cm}
			\begin{tikzpicture}[scale=0.9,every node/.style={transform shape}]
	\node [output_neuron] (out0) at (-3,12)  {} ;

	\node (input0) at (-4,8)  {$x_1$};
	\node (input1) at (-2,8) {$x_2$};

	\node [hidden_neuron] (hi0) at (-6,10)  {  } ;
	\node [hidden_neuron] (hi1) at (-4,10)  {  } ;
	\node [hidden_neuron] (hi2) at (-2,10)  {  } ;
	\node [hidden_neuron] (hi3) at (0,10)  {  } ;

	\node (h1_label) at (-6,10.75)  { $h_1$ } ;
	\node (h2_label) at (-4,10.75)  { $h_2$ } ;
	\node (h3_label) at (-2,10.75)  { $h_3$ } ;
	\node (h4_label) at (-0,10.75)  { $h_4$ } ;

	\node [hidden_neuron] (hi0) at (-6,10)  { -1,-1 } ;
	\node [hidden_neuron] (hi1) at (-4,10)  { -1,1 } ;
	\node [hidden_neuron] (hi2) at (-2,10)  { 1,-1 } ;
	\node [hidden_neuron] (hi3) at (0,10)  { 1,1 } ;


	\node (bias) at (-6,9)  {$bias=$-2} ;

	\node (output0) at (-3,13)  {$y$} ;

	\node (w1) at (-5,11)  {$w_1$} ;
	\node (w2) at (-3.8,11)  {$w_2$} ;
	\node (w3) at (-2.8,11)  {$w_3$} ;
	\node (w4) at (-1.85,11)  {$w_4$} ;

	\draw [->, thick, red] (input0) -- (hi0);
	\draw [->, thick, red] (input1) -- (hi0);

	\draw [->, thick, red](input0) -- (hi1);
	\draw [->, thick, blue](input1) -- (hi1);

	\draw [->, thick, blue] (input0) -- (hi2);
	\draw [->, thick, red] (input1) -- (hi2);

	\draw [->, thick, blue](input0) -- (hi3);
	\draw [->, thick, blue](input1) -- (hi3);

	\draw [->] (hi0) -- (out0);
	\draw [->] (hi1) -- (out0);
	\draw [->] (hi2) -- (out0);
	\draw [->] (hi3) -- (out0);

	\draw [->] (out0) -- (output0);
\end{tikzpicture}

			red edge indicates $w$ = -1 \\
			blue edge indicates $w$ = +1
		\end{overlayarea}
		\column{0.55\textwidth}
		\begin{overlayarea}{\textwidth}{\textheight}
			\begin{itemize}\justifying
				\item<1-> It should be clear that the same network can be used to represent the remaining 15 boolean functions also
				\item<2-> Each boolean function will result in a different set of non-contradicting inequalities which can be satisfied by appropriately setting $w_1, w_2, w_3, w_4$
				\item<3-> Try it!
			\end{itemize}
		\end{overlayarea}
	\end{columns}
\end{frame}

\begin{frame}
	\begin{itemize}\justifying
		\item<1-> What if we have more than 3 inputs ?
	\end{itemize}
\end{frame}

\begin{frame}
	\begin{itemize}\justifying
		\item Again each of the 8 perceptorns will fire only for one of the 8 inputs
		\item Each of the 8 weights in the second layer is responsible for one of the 8 inputs and can be adjusted to produce the desired output for that input
	\end{itemize}
	\begin{tikzpicture}[scale=0.9,every node/.style={transform shape}]
	\node [output_neuron] (out0) at (0,12)  {} ;

	\node (input0) at (-0.75,7)  {$x_1$};
	\node (input1) at (0.75,7) {$x_2$};
	\node (input2) at (2.25,7) {$x_3$};

	\node [hidden_neuron] (hi0) at (-4.5,10)  {  } ;
	\node [hidden_neuron] (hi1) at (-3,10)  {  } ;
	\node [hidden_neuron] (hi2) at (-1.5,10)  {  } ;
	\node [hidden_neuron] (hi3) at (0,10)  {  } ;
	\node [hidden_neuron] (hi4) at (1.5,10)  {  } ;
	\node [hidden_neuron] (hi5) at (3,10)  {  } ;
	\node [hidden_neuron] (hi6) at (4.5,10)  {  } ;
	\node [hidden_neuron] (hi7) at (6,10)  {  } ;



	\node (bias) at (-6,10)  {$bias=$-3} ;

	\node (output0) at (0,13)  {$y$} ;

	\node (w1) at (-2.8,11)  {$w_1$} ;
	\node (w2) at (-1.85,11)  {$w_2$} ;
	\node (w3) at (-1.0,11)  {$w_3$} ;
	\node (w3) at (-0.3,11)  {$w_4$} ;
	\node (w3) at (0.4,11)  {$w_5$} ;
	\node (w3) at (1.2,11)  {$w_6$} ;
	\node (w4) at (1.85,11)  {$w_7$} ;
	\node (w4) at (2.7,11)  {$w_8$} ;

	\draw [->,  red] (input0) -- (hi0);
	\draw [->,  red] (input1) -- (hi0);
	\draw [->,  red] (input2) -- (hi0);

	\draw [->,  red](input0) -- (hi1);
	\draw [->,  red](input1) -- (hi1);
	\draw [->,  blue] (input2) -- (hi1);

	\draw [->,  red] (input0) -- (hi2);
	\draw [->,  blue] (input1) -- (hi2);
	\draw [->,  red] (input2) -- (hi2);

	\draw [->,  red](input0) -- (hi3);
	\draw [->,  blue](input1) -- (hi3);
	\draw [->,  blue] (input2) -- (hi3);

	\draw [->,  blue](input0) -- (hi4);
	\draw [->,  red](input1) -- (hi4);
	\draw [->,  red] (input2) -- (hi4);

	\draw [->,  blue](input0) -- (hi5);
	\draw [->,  red](input1) -- (hi5);
	\draw [->,  blue] (input2) -- (hi5);

	\draw [->,  blue](input0) -- (hi6);
	\draw [->,  blue](input1) -- (hi6);
	\draw [->,  red] (input2) -- (hi6);

	\draw [->,  blue](input0) -- (hi7);
	\draw [->,  blue](input1) -- (hi7);
	\draw [->,  blue] (input2) -- (hi7);

	\draw [->] (hi0) -- (out0);
	\draw [->] (hi1) -- (out0);
	\draw [->] (hi2) -- (out0);
	\draw [->] (hi3) -- (out0);
	\draw [->] (hi4) -- (out0);
	\draw [->] (hi5) -- (out0);
	\draw [->] (hi6) -- (out0);
	\draw [->] (hi7) -- (out0);

	\draw [->] (out0) -- (output0);

	%\node at (-6,1) {red edge indicates $w$ = -1};
	%\node at (-6,0) {green edge indicates $w$ = +1};

\end{tikzpicture}

\end{frame}

\begin{frame}
	\begin{itemize}\justifying
		\item<1-> What if we have $n$ inputs ?
	\end{itemize}
\end{frame}

\begin{frame}
	\begin{block}{Theorem}
		Any boolean function of $n$ inputs can be represented exactly by a network of perceptrons containing 1 hidden layer with $2^n$ perceptrons and one output layer containing $1$ perceptron \\
		\vspace{0.2in}
		\onslide<2->{\textbf{Proof (informal:)} We just saw how to construct such a network} \\
		\vspace{0.2in}
		\onslide<2->{\textbf{Note:} A network of $2^n + 1$ perceptrons is not necessary but sufficient. For example, we already saw how to represent AND function with just 1 perceptron}\\
		\vspace{0.2in}
		\onslide<3->{\textbf{Catch:} As $n$ increases the number of perceptrons in the hidden layers obviously increases exponentially}
	\end{block}
\end{frame}


\begin{frame}
	\begin{itemize}\justifying
		\item<1->{Again, why do we care about boolean functions ?}
		\item<2->{How does this help us with our original problem: which was to predict whether we like a movie or not? \onslide<3-> Let us see!}
	\end{itemize}
\end{frame}

\begin{frame}
	\begin{columns}
		\column{0.4\textwidth}
		\begin{overlayarea}{\textwidth}{\textheight}
			\onslide<5->{
			\begin{center}
				\begin{tikzpicture}[scale=0.4,every node/.style={transform shape}]
	\node [output_neuron] (out0) at (0,12)  {} ;

	\node (input0) at (-0.75,7)  {$x_1$};
	\node (input1) at (0.75,7) {$x_2$};
	\node (input2) at (2.25,7) {$x_3$};

	\node [hidden_neuron] (hi0) at (-4.5,10)  {  } ;
	\node [hidden_neuron] (hi1) at (-3,10)  {  } ;
	\node [hidden_neuron] (hi2) at (-1.5,10)  {  } ;
	\node [hidden_neuron] (hi3) at (0,10)  {  } ;
	\node [hidden_neuron] (hi4) at (1.5,10)  {  } ;
	\node [hidden_neuron] (hi5) at (3,10)  {  } ;
	\node [hidden_neuron] (hi6) at (4.5,10)  {  } ;
	\node [hidden_neuron] (hi7) at (6,10)  {  } ;



	\node (bias) at (-6,10)  {$bias=$-3} ;

	\node (output0) at (0,13)  {$y$} ;

	\node (w1) at (-2.8,11)  {$w_1$} ;
	\node (w2) at (-1.85,11)  {$w_2$} ;
	\node (w3) at (-1.0,11)  {$w_3$} ;
	\node (w3) at (-0.3,11)  {$w_4$} ;
	\node (w3) at (0.4,11)  {$w_5$} ;
	\node (w3) at (1.2,11)  {$w_6$} ;
	\node (w4) at (1.85,11)  {$w_7$} ;
	\node (w4) at (2.7,11)  {$w_8$} ;

	\draw [->,  red] (input0) -- (hi0);
	\draw [->,  red] (input1) -- (hi0);
	\draw [->,  red] (input2) -- (hi0);

	\draw [->,  red](input0) -- (hi1);
	\draw [->,  red](input1) -- (hi1);
	\draw [->,  blue] (input2) -- (hi1);

	\draw [->,  red] (input0) -- (hi2);
	\draw [->,  blue] (input1) -- (hi2);
	\draw [->,  red] (input2) -- (hi2);

	\draw [->,  red](input0) -- (hi3);
	\draw [->,  blue](input1) -- (hi3);
	\draw [->,  blue] (input2) -- (hi3);

	\draw [->,  blue](input0) -- (hi4);
	\draw [->,  red](input1) -- (hi4);
	\draw [->,  red] (input2) -- (hi4);

	\draw [->,  blue](input0) -- (hi5);
	\draw [->,  red](input1) -- (hi5);
	\draw [->,  blue] (input2) -- (hi5);

	\draw [->,  blue](input0) -- (hi6);
	\draw [->,  blue](input1) -- (hi6);
	\draw [->,  red] (input2) -- (hi6);

	\draw [->,  blue](input0) -- (hi7);
	\draw [->,  blue](input1) -- (hi7);
	\draw [->,  blue] (input2) -- (hi7);

	\draw [->] (hi0) -- (out0);
	\draw [->] (hi1) -- (out0);
	\draw [->] (hi2) -- (out0);
	\draw [->] (hi3) -- (out0);
	\draw [->] (hi4) -- (out0);
	\draw [->] (hi5) -- (out0);
	\draw [->] (hi6) -- (out0);
	\draw [->] (hi7) -- (out0);

	\draw [->] (out0) -- (output0);

	%\node at (-6,1) {red edge indicates $w$ = -1};
	%\node at (-6,0) {green edge indicates $w$ = +1};

\end{tikzpicture}

			\end{center}}
			\begin{align*}
				\begin{array}{c} p_1 \\ p_2 \\ \vdots \\ n_1 \\ n_2 \\ \vdots\\ \end{array} &
				\begin{bmatrix}
					x_{11} & x_{12} & \dots  & x_{1n} & y_1=1  \\
					x_{21} & x_{22} & \dots  & x_{2n} & y_2=1  \\
					\vdots & \vdots & \vdots & \vdots & \vdots \\
					x_{k1} & x_{k2} & \dots  & x_{kn} & y_i=0  \\
					x_{j1} & x_{j2} & \dots  & x_{jn} & y_j=0  \\
					\vdots & \vdots & \vdots & \vdots & \vdots \\
				\end{bmatrix}
			\end{align*}
		\end{overlayarea}
		\column{0.6\textwidth}
		\begin{overlayarea}{\textwidth}{\textheight}
			\begin{itemize}\justifying
				\item<1-> We are given this data about our past movie experience
				\item<2->For each movie, we are given the values of the various factors $(x_1, x_2,\dots, x_n)$ that we base our decision on and we are also also given the value of $y$ (like/dislike)
				\item<3-> $p_i$'s are the points for which the output was $1$ and $n_i$'s are the points for which it was $0$
				\item<4->The data may or may not be linearly separable
				\item<5-> The proof that we just saw tells us that it is possible to have a network of perceptrons and learn the weights in this network such that for any given $p_i$ or $n_j$ the output of the network will be the same as $y_i$ or $y_j$ (i.e., we can separate the positive and the negative points)
			\end{itemize}
		\end{overlayarea}
	\end{columns}
\end{frame}

\begin{frame}
	\begin{block}{The story so far ...}
		\begin{itemize}\justifying
			\item<1-> Networks of the form that we just saw (containing, an input, output and one or more hidden layers) are called Multilayer Perceptrons (MLP, in short)
			\item<2-> More appropriate terminology would be``Multilayered Network of Perceptrons'' but MLP is the more commonly used name
			\item<3-> The theorem that we just saw gives us the representation power of a MLP with a single hidden layer
			\item<4-> Specifically, it tells us that a MLP with a single hidden layer can represent \textbf{any} boolean function
		\end{itemize}
	\end{block}
\end{frame}

